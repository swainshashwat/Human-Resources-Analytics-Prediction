
% Default to the notebook output style

    


% Inherit from the specified cell style.




    
\documentclass[11pt]{article}

    
    
    \usepackage[T1]{fontenc}
    % Nicer default font (+ math font) than Computer Modern for most use cases
    \usepackage{mathpazo}

    % Basic figure setup, for now with no caption control since it's done
    % automatically by Pandoc (which extracts ![](path) syntax from Markdown).
    \usepackage{graphicx}
    % We will generate all images so they have a width \maxwidth. This means
    % that they will get their normal width if they fit onto the page, but
    % are scaled down if they would overflow the margins.
    \makeatletter
    \def\maxwidth{\ifdim\Gin@nat@width>\linewidth\linewidth
    \else\Gin@nat@width\fi}
    \makeatother
    \let\Oldincludegraphics\includegraphics
    % Set max figure width to be 80% of text width, for now hardcoded.
    \renewcommand{\includegraphics}[1]{\Oldincludegraphics[width=.8\maxwidth]{#1}}
    % Ensure that by default, figures have no caption (until we provide a
    % proper Figure object with a Caption API and a way to capture that
    % in the conversion process - todo).
    \usepackage{caption}
    \DeclareCaptionLabelFormat{nolabel}{}
    \captionsetup{labelformat=nolabel}

    \usepackage{adjustbox} % Used to constrain images to a maximum size 
    \usepackage{xcolor} % Allow colors to be defined
    \usepackage{enumerate} % Needed for markdown enumerations to work
    \usepackage{geometry} % Used to adjust the document margins
    \usepackage{amsmath} % Equations
    \usepackage{amssymb} % Equations
    \usepackage{textcomp} % defines textquotesingle
    % Hack from http://tex.stackexchange.com/a/47451/13684:
    \AtBeginDocument{%
        \def\PYZsq{\textquotesingle}% Upright quotes in Pygmentized code
    }
    \usepackage{upquote} % Upright quotes for verbatim code
    \usepackage{eurosym} % defines \euro
    \usepackage[mathletters]{ucs} % Extended unicode (utf-8) support
    \usepackage[utf8x]{inputenc} % Allow utf-8 characters in the tex document
    \usepackage{fancyvrb} % verbatim replacement that allows latex
    \usepackage{grffile} % extends the file name processing of package graphics 
                         % to support a larger range 
    % The hyperref package gives us a pdf with properly built
    % internal navigation ('pdf bookmarks' for the table of contents,
    % internal cross-reference links, web links for URLs, etc.)
    \usepackage{hyperref}
    \usepackage{longtable} % longtable support required by pandoc >1.10
    \usepackage{booktabs}  % table support for pandoc > 1.12.2
    \usepackage[inline]{enumitem} % IRkernel/repr support (it uses the enumerate* environment)
    \usepackage[normalem]{ulem} % ulem is needed to support strikethroughs (\sout)
                                % normalem makes italics be italics, not underlines
    

    
    
    % Colors for the hyperref package
    \definecolor{urlcolor}{rgb}{0,.145,.698}
    \definecolor{linkcolor}{rgb}{.71,0.21,0.01}
    \definecolor{citecolor}{rgb}{.12,.54,.11}

    % ANSI colors
    \definecolor{ansi-black}{HTML}{3E424D}
    \definecolor{ansi-black-intense}{HTML}{282C36}
    \definecolor{ansi-red}{HTML}{E75C58}
    \definecolor{ansi-red-intense}{HTML}{B22B31}
    \definecolor{ansi-green}{HTML}{00A250}
    \definecolor{ansi-green-intense}{HTML}{007427}
    \definecolor{ansi-yellow}{HTML}{DDB62B}
    \definecolor{ansi-yellow-intense}{HTML}{B27D12}
    \definecolor{ansi-blue}{HTML}{208FFB}
    \definecolor{ansi-blue-intense}{HTML}{0065CA}
    \definecolor{ansi-magenta}{HTML}{D160C4}
    \definecolor{ansi-magenta-intense}{HTML}{A03196}
    \definecolor{ansi-cyan}{HTML}{60C6C8}
    \definecolor{ansi-cyan-intense}{HTML}{258F8F}
    \definecolor{ansi-white}{HTML}{C5C1B4}
    \definecolor{ansi-white-intense}{HTML}{A1A6B2}

    % commands and environments needed by pandoc snippets
    % extracted from the output of `pandoc -s`
    \providecommand{\tightlist}{%
      \setlength{\itemsep}{0pt}\setlength{\parskip}{0pt}}
    \DefineVerbatimEnvironment{Highlighting}{Verbatim}{commandchars=\\\{\}}
    % Add ',fontsize=\small' for more characters per line
    \newenvironment{Shaded}{}{}
    \newcommand{\KeywordTok}[1]{\textcolor[rgb]{0.00,0.44,0.13}{\textbf{{#1}}}}
    \newcommand{\DataTypeTok}[1]{\textcolor[rgb]{0.56,0.13,0.00}{{#1}}}
    \newcommand{\DecValTok}[1]{\textcolor[rgb]{0.25,0.63,0.44}{{#1}}}
    \newcommand{\BaseNTok}[1]{\textcolor[rgb]{0.25,0.63,0.44}{{#1}}}
    \newcommand{\FloatTok}[1]{\textcolor[rgb]{0.25,0.63,0.44}{{#1}}}
    \newcommand{\CharTok}[1]{\textcolor[rgb]{0.25,0.44,0.63}{{#1}}}
    \newcommand{\StringTok}[1]{\textcolor[rgb]{0.25,0.44,0.63}{{#1}}}
    \newcommand{\CommentTok}[1]{\textcolor[rgb]{0.38,0.63,0.69}{\textit{{#1}}}}
    \newcommand{\OtherTok}[1]{\textcolor[rgb]{0.00,0.44,0.13}{{#1}}}
    \newcommand{\AlertTok}[1]{\textcolor[rgb]{1.00,0.00,0.00}{\textbf{{#1}}}}
    \newcommand{\FunctionTok}[1]{\textcolor[rgb]{0.02,0.16,0.49}{{#1}}}
    \newcommand{\RegionMarkerTok}[1]{{#1}}
    \newcommand{\ErrorTok}[1]{\textcolor[rgb]{1.00,0.00,0.00}{\textbf{{#1}}}}
    \newcommand{\NormalTok}[1]{{#1}}
    
    % Additional commands for more recent versions of Pandoc
    \newcommand{\ConstantTok}[1]{\textcolor[rgb]{0.53,0.00,0.00}{{#1}}}
    \newcommand{\SpecialCharTok}[1]{\textcolor[rgb]{0.25,0.44,0.63}{{#1}}}
    \newcommand{\VerbatimStringTok}[1]{\textcolor[rgb]{0.25,0.44,0.63}{{#1}}}
    \newcommand{\SpecialStringTok}[1]{\textcolor[rgb]{0.73,0.40,0.53}{{#1}}}
    \newcommand{\ImportTok}[1]{{#1}}
    \newcommand{\DocumentationTok}[1]{\textcolor[rgb]{0.73,0.13,0.13}{\textit{{#1}}}}
    \newcommand{\AnnotationTok}[1]{\textcolor[rgb]{0.38,0.63,0.69}{\textbf{\textit{{#1}}}}}
    \newcommand{\CommentVarTok}[1]{\textcolor[rgb]{0.38,0.63,0.69}{\textbf{\textit{{#1}}}}}
    \newcommand{\VariableTok}[1]{\textcolor[rgb]{0.10,0.09,0.49}{{#1}}}
    \newcommand{\ControlFlowTok}[1]{\textcolor[rgb]{0.00,0.44,0.13}{\textbf{{#1}}}}
    \newcommand{\OperatorTok}[1]{\textcolor[rgb]{0.40,0.40,0.40}{{#1}}}
    \newcommand{\BuiltInTok}[1]{{#1}}
    \newcommand{\ExtensionTok}[1]{{#1}}
    \newcommand{\PreprocessorTok}[1]{\textcolor[rgb]{0.74,0.48,0.00}{{#1}}}
    \newcommand{\AttributeTok}[1]{\textcolor[rgb]{0.49,0.56,0.16}{{#1}}}
    \newcommand{\InformationTok}[1]{\textcolor[rgb]{0.38,0.63,0.69}{\textbf{\textit{{#1}}}}}
    \newcommand{\WarningTok}[1]{\textcolor[rgb]{0.38,0.63,0.69}{\textbf{\textit{{#1}}}}}
    
    
    % Define a nice break command that doesn't care if a line doesn't already
    % exist.
    \def\br{\hspace*{\fill} \\* }
    % Math Jax compatability definitions
    \def\gt{>}
    \def\lt{<}
    % Document parameters
    \title{hr\_analytics\_kaggle}
    
    
    

    % Pygments definitions
    
\makeatletter
\def\PY@reset{\let\PY@it=\relax \let\PY@bf=\relax%
    \let\PY@ul=\relax \let\PY@tc=\relax%
    \let\PY@bc=\relax \let\PY@ff=\relax}
\def\PY@tok#1{\csname PY@tok@#1\endcsname}
\def\PY@toks#1+{\ifx\relax#1\empty\else%
    \PY@tok{#1}\expandafter\PY@toks\fi}
\def\PY@do#1{\PY@bc{\PY@tc{\PY@ul{%
    \PY@it{\PY@bf{\PY@ff{#1}}}}}}}
\def\PY#1#2{\PY@reset\PY@toks#1+\relax+\PY@do{#2}}

\expandafter\def\csname PY@tok@w\endcsname{\def\PY@tc##1{\textcolor[rgb]{0.73,0.73,0.73}{##1}}}
\expandafter\def\csname PY@tok@c\endcsname{\let\PY@it=\textit\def\PY@tc##1{\textcolor[rgb]{0.25,0.50,0.50}{##1}}}
\expandafter\def\csname PY@tok@cp\endcsname{\def\PY@tc##1{\textcolor[rgb]{0.74,0.48,0.00}{##1}}}
\expandafter\def\csname PY@tok@k\endcsname{\let\PY@bf=\textbf\def\PY@tc##1{\textcolor[rgb]{0.00,0.50,0.00}{##1}}}
\expandafter\def\csname PY@tok@kp\endcsname{\def\PY@tc##1{\textcolor[rgb]{0.00,0.50,0.00}{##1}}}
\expandafter\def\csname PY@tok@kt\endcsname{\def\PY@tc##1{\textcolor[rgb]{0.69,0.00,0.25}{##1}}}
\expandafter\def\csname PY@tok@o\endcsname{\def\PY@tc##1{\textcolor[rgb]{0.40,0.40,0.40}{##1}}}
\expandafter\def\csname PY@tok@ow\endcsname{\let\PY@bf=\textbf\def\PY@tc##1{\textcolor[rgb]{0.67,0.13,1.00}{##1}}}
\expandafter\def\csname PY@tok@nb\endcsname{\def\PY@tc##1{\textcolor[rgb]{0.00,0.50,0.00}{##1}}}
\expandafter\def\csname PY@tok@nf\endcsname{\def\PY@tc##1{\textcolor[rgb]{0.00,0.00,1.00}{##1}}}
\expandafter\def\csname PY@tok@nc\endcsname{\let\PY@bf=\textbf\def\PY@tc##1{\textcolor[rgb]{0.00,0.00,1.00}{##1}}}
\expandafter\def\csname PY@tok@nn\endcsname{\let\PY@bf=\textbf\def\PY@tc##1{\textcolor[rgb]{0.00,0.00,1.00}{##1}}}
\expandafter\def\csname PY@tok@ne\endcsname{\let\PY@bf=\textbf\def\PY@tc##1{\textcolor[rgb]{0.82,0.25,0.23}{##1}}}
\expandafter\def\csname PY@tok@nv\endcsname{\def\PY@tc##1{\textcolor[rgb]{0.10,0.09,0.49}{##1}}}
\expandafter\def\csname PY@tok@no\endcsname{\def\PY@tc##1{\textcolor[rgb]{0.53,0.00,0.00}{##1}}}
\expandafter\def\csname PY@tok@nl\endcsname{\def\PY@tc##1{\textcolor[rgb]{0.63,0.63,0.00}{##1}}}
\expandafter\def\csname PY@tok@ni\endcsname{\let\PY@bf=\textbf\def\PY@tc##1{\textcolor[rgb]{0.60,0.60,0.60}{##1}}}
\expandafter\def\csname PY@tok@na\endcsname{\def\PY@tc##1{\textcolor[rgb]{0.49,0.56,0.16}{##1}}}
\expandafter\def\csname PY@tok@nt\endcsname{\let\PY@bf=\textbf\def\PY@tc##1{\textcolor[rgb]{0.00,0.50,0.00}{##1}}}
\expandafter\def\csname PY@tok@nd\endcsname{\def\PY@tc##1{\textcolor[rgb]{0.67,0.13,1.00}{##1}}}
\expandafter\def\csname PY@tok@s\endcsname{\def\PY@tc##1{\textcolor[rgb]{0.73,0.13,0.13}{##1}}}
\expandafter\def\csname PY@tok@sd\endcsname{\let\PY@it=\textit\def\PY@tc##1{\textcolor[rgb]{0.73,0.13,0.13}{##1}}}
\expandafter\def\csname PY@tok@si\endcsname{\let\PY@bf=\textbf\def\PY@tc##1{\textcolor[rgb]{0.73,0.40,0.53}{##1}}}
\expandafter\def\csname PY@tok@se\endcsname{\let\PY@bf=\textbf\def\PY@tc##1{\textcolor[rgb]{0.73,0.40,0.13}{##1}}}
\expandafter\def\csname PY@tok@sr\endcsname{\def\PY@tc##1{\textcolor[rgb]{0.73,0.40,0.53}{##1}}}
\expandafter\def\csname PY@tok@ss\endcsname{\def\PY@tc##1{\textcolor[rgb]{0.10,0.09,0.49}{##1}}}
\expandafter\def\csname PY@tok@sx\endcsname{\def\PY@tc##1{\textcolor[rgb]{0.00,0.50,0.00}{##1}}}
\expandafter\def\csname PY@tok@m\endcsname{\def\PY@tc##1{\textcolor[rgb]{0.40,0.40,0.40}{##1}}}
\expandafter\def\csname PY@tok@gh\endcsname{\let\PY@bf=\textbf\def\PY@tc##1{\textcolor[rgb]{0.00,0.00,0.50}{##1}}}
\expandafter\def\csname PY@tok@gu\endcsname{\let\PY@bf=\textbf\def\PY@tc##1{\textcolor[rgb]{0.50,0.00,0.50}{##1}}}
\expandafter\def\csname PY@tok@gd\endcsname{\def\PY@tc##1{\textcolor[rgb]{0.63,0.00,0.00}{##1}}}
\expandafter\def\csname PY@tok@gi\endcsname{\def\PY@tc##1{\textcolor[rgb]{0.00,0.63,0.00}{##1}}}
\expandafter\def\csname PY@tok@gr\endcsname{\def\PY@tc##1{\textcolor[rgb]{1.00,0.00,0.00}{##1}}}
\expandafter\def\csname PY@tok@ge\endcsname{\let\PY@it=\textit}
\expandafter\def\csname PY@tok@gs\endcsname{\let\PY@bf=\textbf}
\expandafter\def\csname PY@tok@gp\endcsname{\let\PY@bf=\textbf\def\PY@tc##1{\textcolor[rgb]{0.00,0.00,0.50}{##1}}}
\expandafter\def\csname PY@tok@go\endcsname{\def\PY@tc##1{\textcolor[rgb]{0.53,0.53,0.53}{##1}}}
\expandafter\def\csname PY@tok@gt\endcsname{\def\PY@tc##1{\textcolor[rgb]{0.00,0.27,0.87}{##1}}}
\expandafter\def\csname PY@tok@err\endcsname{\def\PY@bc##1{\setlength{\fboxsep}{0pt}\fcolorbox[rgb]{1.00,0.00,0.00}{1,1,1}{\strut ##1}}}
\expandafter\def\csname PY@tok@kc\endcsname{\let\PY@bf=\textbf\def\PY@tc##1{\textcolor[rgb]{0.00,0.50,0.00}{##1}}}
\expandafter\def\csname PY@tok@kd\endcsname{\let\PY@bf=\textbf\def\PY@tc##1{\textcolor[rgb]{0.00,0.50,0.00}{##1}}}
\expandafter\def\csname PY@tok@kn\endcsname{\let\PY@bf=\textbf\def\PY@tc##1{\textcolor[rgb]{0.00,0.50,0.00}{##1}}}
\expandafter\def\csname PY@tok@kr\endcsname{\let\PY@bf=\textbf\def\PY@tc##1{\textcolor[rgb]{0.00,0.50,0.00}{##1}}}
\expandafter\def\csname PY@tok@bp\endcsname{\def\PY@tc##1{\textcolor[rgb]{0.00,0.50,0.00}{##1}}}
\expandafter\def\csname PY@tok@fm\endcsname{\def\PY@tc##1{\textcolor[rgb]{0.00,0.00,1.00}{##1}}}
\expandafter\def\csname PY@tok@vc\endcsname{\def\PY@tc##1{\textcolor[rgb]{0.10,0.09,0.49}{##1}}}
\expandafter\def\csname PY@tok@vg\endcsname{\def\PY@tc##1{\textcolor[rgb]{0.10,0.09,0.49}{##1}}}
\expandafter\def\csname PY@tok@vi\endcsname{\def\PY@tc##1{\textcolor[rgb]{0.10,0.09,0.49}{##1}}}
\expandafter\def\csname PY@tok@vm\endcsname{\def\PY@tc##1{\textcolor[rgb]{0.10,0.09,0.49}{##1}}}
\expandafter\def\csname PY@tok@sa\endcsname{\def\PY@tc##1{\textcolor[rgb]{0.73,0.13,0.13}{##1}}}
\expandafter\def\csname PY@tok@sb\endcsname{\def\PY@tc##1{\textcolor[rgb]{0.73,0.13,0.13}{##1}}}
\expandafter\def\csname PY@tok@sc\endcsname{\def\PY@tc##1{\textcolor[rgb]{0.73,0.13,0.13}{##1}}}
\expandafter\def\csname PY@tok@dl\endcsname{\def\PY@tc##1{\textcolor[rgb]{0.73,0.13,0.13}{##1}}}
\expandafter\def\csname PY@tok@s2\endcsname{\def\PY@tc##1{\textcolor[rgb]{0.73,0.13,0.13}{##1}}}
\expandafter\def\csname PY@tok@sh\endcsname{\def\PY@tc##1{\textcolor[rgb]{0.73,0.13,0.13}{##1}}}
\expandafter\def\csname PY@tok@s1\endcsname{\def\PY@tc##1{\textcolor[rgb]{0.73,0.13,0.13}{##1}}}
\expandafter\def\csname PY@tok@mb\endcsname{\def\PY@tc##1{\textcolor[rgb]{0.40,0.40,0.40}{##1}}}
\expandafter\def\csname PY@tok@mf\endcsname{\def\PY@tc##1{\textcolor[rgb]{0.40,0.40,0.40}{##1}}}
\expandafter\def\csname PY@tok@mh\endcsname{\def\PY@tc##1{\textcolor[rgb]{0.40,0.40,0.40}{##1}}}
\expandafter\def\csname PY@tok@mi\endcsname{\def\PY@tc##1{\textcolor[rgb]{0.40,0.40,0.40}{##1}}}
\expandafter\def\csname PY@tok@il\endcsname{\def\PY@tc##1{\textcolor[rgb]{0.40,0.40,0.40}{##1}}}
\expandafter\def\csname PY@tok@mo\endcsname{\def\PY@tc##1{\textcolor[rgb]{0.40,0.40,0.40}{##1}}}
\expandafter\def\csname PY@tok@ch\endcsname{\let\PY@it=\textit\def\PY@tc##1{\textcolor[rgb]{0.25,0.50,0.50}{##1}}}
\expandafter\def\csname PY@tok@cm\endcsname{\let\PY@it=\textit\def\PY@tc##1{\textcolor[rgb]{0.25,0.50,0.50}{##1}}}
\expandafter\def\csname PY@tok@cpf\endcsname{\let\PY@it=\textit\def\PY@tc##1{\textcolor[rgb]{0.25,0.50,0.50}{##1}}}
\expandafter\def\csname PY@tok@c1\endcsname{\let\PY@it=\textit\def\PY@tc##1{\textcolor[rgb]{0.25,0.50,0.50}{##1}}}
\expandafter\def\csname PY@tok@cs\endcsname{\let\PY@it=\textit\def\PY@tc##1{\textcolor[rgb]{0.25,0.50,0.50}{##1}}}

\def\PYZbs{\char`\\}
\def\PYZus{\char`\_}
\def\PYZob{\char`\{}
\def\PYZcb{\char`\}}
\def\PYZca{\char`\^}
\def\PYZam{\char`\&}
\def\PYZlt{\char`\<}
\def\PYZgt{\char`\>}
\def\PYZsh{\char`\#}
\def\PYZpc{\char`\%}
\def\PYZdl{\char`\$}
\def\PYZhy{\char`\-}
\def\PYZsq{\char`\'}
\def\PYZdq{\char`\"}
\def\PYZti{\char`\~}
% for compatibility with earlier versions
\def\PYZat{@}
\def\PYZlb{[}
\def\PYZrb{]}
\makeatother


    % Exact colors from NB
    \definecolor{incolor}{rgb}{0.0, 0.0, 0.5}
    \definecolor{outcolor}{rgb}{0.545, 0.0, 0.0}



    
    % Prevent overflowing lines due to hard-to-break entities
    \sloppy 
    % Setup hyperref package
    \hypersetup{
      breaklinks=true,  % so long urls are correctly broken across lines
      colorlinks=true,
      urlcolor=urlcolor,
      linkcolor=linkcolor,
      citecolor=citecolor,
      }
    % Slightly bigger margins than the latex defaults
    
    \geometry{verbose,tmargin=1in,bmargin=1in,lmargin=1in,rmargin=1in}
    
    

    \begin{document}
    
    
    \maketitle
    
    

    
    \subsubsection{Pandas}\label{pandas}

\textbf{\href{https://pandas.pydata.org/pandas-docs/version/0.20/}{Pandas}}
is a Python package providing fast, flexible, and expressive data
structures designed to make working with structured (tabular,
multidimensional, potentially heterogeneous) and time series data both
easy and intuitive.

\subsubsection{Numpy}\label{numpy}

\textbf{\href{https://docs.scipy.org/doc/numpy/user/quickstart.html}{Numpy}}
is the fundamental package for scientific computing with Python. It
contains among other things:

\begin{itemize}
\tightlist
\item
  powerful N-dimensional array object
\item
  sophisticated (broadcasting) functions
\item
  tools for integrating C/C++ and Fortran code
\item
  useful linear algebra, Fourier transform, and random number
  capabilities
\end{itemize}

\subsubsection{Matplotlib}\label{matplotlib}

\textbf{\href{https://matplotlib.org/contents.html}{Matplotlib}} is a
Python 2D plotting library which produces publication quality figures in
a variety of hardcopy formats and interactive environments across
platforms. Matplotlib can be used in Python scripts, the Python and
IPython shells, the Jupyter notebook, web application servers, and four
graphical user interface toolkits.

\subsubsection{Seaborn}\label{seaborn}

\textbf{\href{https://seaborn.pydata.org/tutorial.html}{Seaborn}} is a
Python visualization library based on matplotlib. It provides a
high-level interface for drawing attractive statistical graphics.

    \subsection{Importing the necessary
packages}\label{importing-the-necessary-packages}

    \begin{Verbatim}[commandchars=\\\{\}]
{\color{incolor}In [{\color{incolor}2}]:} \PY{c+c1}{\PYZsh{} importing the basic packages}
        
        \PY{k+kn}{import} \PY{n+nn}{pandas} \PY{k}{as} \PY{n+nn}{pd}
        \PY{k+kn}{import} \PY{n+nn}{numpy} \PY{k}{as} \PY{n+nn}{np}
        \PY{k+kn}{import} \PY{n+nn}{seaborn} \PY{k}{as} \PY{n+nn}{sns}
        \PY{k+kn}{import} \PY{n+nn}{matplotlib}\PY{n+nn}{.}\PY{n+nn}{pyplot} \PY{k}{as} \PY{n+nn}{plt}
\end{Verbatim}


    \textbf{\href{http://ipython.readthedocs.io/en/stable/interactive/tutorial.html\#magics-explained}{Magic
Commands}}

    \begin{Verbatim}[commandchars=\\\{\}]
{\color{incolor}In [{\color{incolor}4}]:} \PY{c+c1}{\PYZsh{} Special line in jupyter notebooks that helps to display the plot in the notebook itself}
        
        \PY{o}{\PYZpc{}}\PY{k}{matplotlib} inline
\end{Verbatim}


    \section{Data Preprocessing}\label{data-preprocessing}

    \begin{Verbatim}[commandchars=\\\{\}]
{\color{incolor}In [{\color{incolor}5}]:} \PY{c+c1}{\PYZsh{} importing the datafile}
        \PY{n}{df} \PY{o}{=} \PY{n}{pd}\PY{o}{.}\PY{n}{read\PYZus{}csv}\PY{p}{(}\PY{l+s+s1}{\PYZsq{}}\PY{l+s+s1}{turnover.csv}\PY{l+s+s1}{\PYZsq{}}\PY{p}{)}
\end{Verbatim}


    \subsubsection{Understanding the data}\label{understanding-the-data}

    \begin{quote}
DataFrame.info(verbose=None, buf=None, max\_cols=None,
memory\_usage=None, null\_counts=None)
\end{quote}

Print a concise summary of a DataFrame.

    \begin{Verbatim}[commandchars=\\\{\}]
{\color{incolor}In [{\color{incolor}13}]:} \PY{c+c1}{\PYZsh{} Checking for NULL values}
         \PY{c+c1}{\PYZsh{} From the results, it\PYZsq{}s clear that the dataset does NOT contain any null/nan/empty values}
         \PY{n}{df}\PY{o}{.}\PY{n}{info}\PY{p}{(}\PY{p}{)}
\end{Verbatim}


    \begin{Verbatim}[commandchars=\\\{\}]
<class 'pandas.core.frame.DataFrame'>
RangeIndex: 14999 entries, 0 to 14998
Data columns (total 10 columns):
satisfaction\_level       14999 non-null float64
last\_evaluation          14999 non-null float64
number\_project           14999 non-null int64
average\_montly\_hours     14999 non-null int64
time\_spend\_company       14999 non-null int64
Work\_accident            14999 non-null int64
left                     14999 non-null int64
promotion\_last\_5years    14999 non-null int64
sales                    14999 non-null object
salary                   14999 non-null object
dtypes: float64(2), int64(6), object(2)
memory usage: 1.1+ MB

    \end{Verbatim}

    \begin{quote}
DataFrame.describe(percentiles=None, include=None, exclude=None)
\end{quote}

Generates descriptive statistics that summarize the central tendency,
dispersion and shape of a dataset's distribution, excluding NaN values.

    \begin{Verbatim}[commandchars=\\\{\}]
{\color{incolor}In [{\color{incolor}14}]:} \PY{c+c1}{\PYZsh{} Checking the distribution of the numerical attributes}
         \PY{n}{df}\PY{o}{.}\PY{n}{describe}\PY{p}{(}\PY{p}{)}
\end{Verbatim}


\begin{Verbatim}[commandchars=\\\{\}]
{\color{outcolor}Out[{\color{outcolor}14}]:}        satisfaction\_level  last\_evaluation  number\_project  \textbackslash{}
         count        14999.000000     14999.000000    14999.000000   
         mean             0.612834         0.716102        3.803054   
         std              0.248631         0.171169        1.232592   
         min              0.090000         0.360000        2.000000   
         25\%              0.440000         0.560000        3.000000   
         50\%              0.640000         0.720000        4.000000   
         75\%              0.820000         0.870000        5.000000   
         max              1.000000         1.000000        7.000000   
         
                average\_montly\_hours  time\_spend\_company  Work\_accident          left  \textbackslash{}
         count          14999.000000        14999.000000   14999.000000  14999.000000   
         mean             201.050337            3.498233       0.144610      0.238083   
         std               49.943099            1.460136       0.351719      0.425924   
         min               96.000000            2.000000       0.000000      0.000000   
         25\%              156.000000            3.000000       0.000000      0.000000   
         50\%              200.000000            3.000000       0.000000      0.000000   
         75\%              245.000000            4.000000       0.000000      0.000000   
         max              310.000000           10.000000       1.000000      1.000000   
         
                promotion\_last\_5years  
         count           14999.000000  
         mean                0.021268  
         std                 0.144281  
         min                 0.000000  
         25\%                 0.000000  
         50\%                 0.000000  
         75\%                 0.000000  
         max                 1.000000  
\end{Verbatim}
            
    \begin{Verbatim}[commandchars=\\\{\}]
{\color{incolor}In [{\color{incolor}15}]:} \PY{n}{df}\PY{o}{.}\PY{n}{head}\PY{p}{(}\PY{p}{)}
\end{Verbatim}


\begin{Verbatim}[commandchars=\\\{\}]
{\color{outcolor}Out[{\color{outcolor}15}]:}    satisfaction\_level  last\_evaluation  number\_project  average\_montly\_hours  \textbackslash{}
         0                0.38             0.53               2                   157   
         1                0.80             0.86               5                   262   
         2                0.11             0.88               7                   272   
         3                0.72             0.87               5                   223   
         4                0.37             0.52               2                   159   
         
            time\_spend\_company  Work\_accident  left  promotion\_last\_5years  sales  \textbackslash{}
         0                   3              0     1                      0  sales   
         1                   6              0     1                      0  sales   
         2                   4              0     1                      0  sales   
         3                   5              0     1                      0  sales   
         4                   3              0     1                      0  sales   
         
            salary  
         0     low  
         1  medium  
         2  medium  
         3     low  
         4     low  
\end{Verbatim}
            
    \begin{Verbatim}[commandchars=\\\{\}]
{\color{incolor}In [{\color{incolor}16}]:} \PY{c+c1}{\PYZsh{} Checking the names of the columns}
         \PY{n}{df}\PY{o}{.}\PY{n}{columns}
\end{Verbatim}


\begin{Verbatim}[commandchars=\\\{\}]
{\color{outcolor}Out[{\color{outcolor}16}]:} Index(['satisfaction\_level', 'last\_evaluation', 'number\_project',
                'average\_montly\_hours', 'time\_spend\_company', 'Work\_accident', 'left',
                'promotion\_last\_5years', 'sales', 'salary'],
               dtype='object')
\end{Verbatim}
            
    \begin{Verbatim}[commandchars=\\\{\}]
{\color{incolor}In [{\color{incolor}17}]:} \PY{c+c1}{\PYZsh{} Checking the values in the output label}
         \PY{n}{df}\PY{o}{.}\PY{n}{left}\PY{o}{.}\PY{n}{unique}\PY{p}{(}\PY{p}{)}
\end{Verbatim}


\begin{Verbatim}[commandchars=\\\{\}]
{\color{outcolor}Out[{\color{outcolor}17}]:} array([1, 0], dtype=int64)
\end{Verbatim}
            
    \subsubsection{Encoding 'sales' and
'salary'}\label{encoding-sales-and-salary}

As the columns \textbf{sales} and \textbf{salary} are non-numerical
column (Object type), we convert them into numerical attributes.

    \paragraph{sales}\label{sales}

    \begin{Verbatim}[commandchars=\\\{\}]
{\color{incolor}In [{\color{incolor}48}]:} \PY{c+c1}{\PYZsh{} Checking the unique values in the \PYZsq{}salary\PYZsq{} column}
         \PY{n}{df}\PY{o}{.}\PY{n}{sales}\PY{o}{.}\PY{n}{unique}\PY{p}{(}\PY{p}{)}
\end{Verbatim}


\begin{Verbatim}[commandchars=\\\{\}]
{\color{outcolor}Out[{\color{outcolor}48}]:} array(['sales', 'accounting', 'hr', 'technical', 'support', 'management',
                'IT', 'product\_mng', 'marketing', 'RandD'], dtype=object)
\end{Verbatim}
            
    \begin{Verbatim}[commandchars=\\\{\}]
{\color{incolor}In [{\color{incolor}49}]:} \PY{c+c1}{\PYZsh{} counting the number unique values in the \PYZsq{}sales\PYZsq{} column}
         \PY{n}{sales\PYZus{}grouped} \PY{o}{=} \PY{n}{pd}\PY{o}{.}\PY{n}{DataFrame}\PY{p}{(}\PY{n}{df}\PY{o}{.}\PY{n}{sales}\PY{o}{.}\PY{n}{groupby}\PY{p}{(}\PY{n}{by}\PY{o}{=}\PY{n}{df}\PY{o}{.}\PY{n}{sales}\PY{p}{)}\PY{o}{.}\PY{n}{count}\PY{p}{(}\PY{p}{)}\PY{p}{)}
         
         \PY{n+nb}{print}\PY{p}{(}\PY{n}{df}\PY{o}{.}\PY{n}{sales}\PY{o}{.}\PY{n}{groupby}\PY{p}{(}\PY{n}{by}\PY{o}{=}\PY{n}{df}\PY{o}{.}\PY{n}{sales}\PY{p}{)}\PY{o}{.}\PY{n}{count}\PY{p}{(}\PY{p}{)}\PY{p}{)}
         \PY{n+nb}{print}\PY{p}{(}\PY{l+s+s2}{\PYZdq{}}\PY{l+s+s2}{\PYZhy{}\PYZhy{}\PYZhy{}\PYZhy{}\PYZhy{}\PYZhy{}\PYZhy{}\PYZhy{}\PYZhy{}\PYZhy{}\PYZhy{}\PYZhy{}\PYZhy{}\PYZhy{}\PYZhy{}\PYZhy{}\PYZhy{}\PYZhy{}\PYZhy{}\PYZhy{}\PYZhy{}\PYZhy{}\PYZhy{}\PYZhy{}\PYZhy{}\PYZhy{}\PYZhy{}\PYZhy{}\PYZhy{}\PYZhy{}\PYZhy{}\PYZhy{}\PYZhy{}}\PY{l+s+s2}{\PYZdq{}}\PY{p}{)}
         \PY{n+nb}{print}\PY{p}{(}\PY{l+s+s2}{\PYZdq{}}\PY{l+s+s2}{Total = }\PY{l+s+s2}{\PYZdq{}}\PY{o}{+} \PY{n+nb}{str}\PY{p}{(}\PY{n}{df}\PY{o}{.}\PY{n}{sales}\PY{o}{.}\PY{n}{count}\PY{p}{(}\PY{p}{)}\PY{p}{)}\PY{p}{)}
\end{Verbatim}


    \begin{Verbatim}[commandchars=\\\{\}]
sales
IT             1227
RandD           787
accounting      767
hr              739
management      630
marketing       858
product\_mng     902
sales          4140
support        2229
technical      2720
Name: sales, dtype: int64
---------------------------------
Total = 14999

    \end{Verbatim}

    \begin{Verbatim}[commandchars=\\\{\}]
{\color{incolor}In [{\color{incolor}84}]:} \PY{c+c1}{\PYZsh{} plotting the grouped unique values in the \PYZsq{}sales\PYZsq{} column}
         \PY{n}{sales\PYZus{}grouped}\PY{o}{.}\PY{n}{plot}\PY{p}{(}\PY{n}{kind}\PY{o}{=}\PY{l+s+s1}{\PYZsq{}}\PY{l+s+s1}{bar}\PY{l+s+s1}{\PYZsq{}}\PY{p}{,} \PY{n}{figsize}\PY{o}{=}\PY{p}{(}\PY{l+m+mi}{15}\PY{p}{,}\PY{l+m+mi}{6}\PY{p}{)}\PY{p}{)}
\end{Verbatim}


\begin{Verbatim}[commandchars=\\\{\}]
{\color{outcolor}Out[{\color{outcolor}84}]:} <matplotlib.axes.\_subplots.AxesSubplot at 0x219211d84e0>
\end{Verbatim}
            
    \begin{center}
    \adjustimage{max size={0.9\linewidth}{0.9\paperheight}}{output_19_1.png}
    \end{center}
    { \hspace*{\fill} \\}
    
    \paragraph{Label Encoding/ Integer
Encoding}\label{label-encoding-integer-encoding}

An approach to encoding categorical values is to use a technique called
label encoding Label encoding is simply converting each value in a
column to a number.

\href{http://scikit-learn.org/stable/modules/generated/sklearn.preprocessing.LabelEncoder.html}{Scikit-learn's
LabelEncoder Documentation}

    \begin{Verbatim}[commandchars=\\\{\}]
{\color{incolor}In [{\color{incolor}51}]:} \PY{k+kn}{from} \PY{n+nn}{sklearn}\PY{n+nn}{.}\PY{n+nn}{preprocessing} \PY{k}{import} \PY{n}{LabelEncoder}
\end{Verbatim}


    \begin{Verbatim}[commandchars=\\\{\}]
{\color{incolor}In [{\color{incolor}52}]:} \PY{c+c1}{\PYZsh{} initializing}
         \PY{n}{label\PYZus{}encoder} \PY{o}{=} \PY{n}{LabelEncoder}\PY{p}{(}\PY{p}{)}
\end{Verbatim}


    \begin{Verbatim}[commandchars=\\\{\}]
{\color{incolor}In [{\color{incolor}59}]:} \PY{n}{sales\PYZus{}label} \PY{o}{=} \PY{n}{label\PYZus{}encoder}\PY{o}{.}\PY{n}{fit\PYZus{}transform}\PY{p}{(}\PY{n}{df}\PY{o}{.}\PY{n}{sales}\PY{p}{)} \PY{c+c1}{\PYZsh{} converting all \PYZsq{}categorical values\PYZsq{} to \PYZsq{}numerical\PYZsq{}}
         
         \PY{n+nb}{print}\PY{p}{(}\PY{n+nb}{type}\PY{p}{(}\PY{n}{sales\PYZus{}label}\PY{p}{[}\PY{l+m+mi}{0}\PY{p}{]}\PY{p}{)}\PY{p}{)} \PY{c+c1}{\PYZsh{} checking the data type for the values in \PYZsq{}sales\PYZus{}label\PYZsq{}}
         
         \PY{n}{sales\PYZus{}label}
\end{Verbatim}


    \begin{Verbatim}[commandchars=\\\{\}]
<class 'numpy.int64'>

    \end{Verbatim}

\begin{Verbatim}[commandchars=\\\{\}]
{\color{outcolor}Out[{\color{outcolor}59}]:} array([7, 7, 7, {\ldots}, 8, 8, 8], dtype=int64)
\end{Verbatim}
            
    \paragraph{salary}\label{salary}

    \begin{Verbatim}[commandchars=\\\{\}]
{\color{incolor}In [{\color{incolor}54}]:} \PY{c+c1}{\PYZsh{} Checking the unique values of the \PYZsq{}salary\PYZsq{} attribute}
         \PY{n}{df}\PY{o}{.}\PY{n}{salary}\PY{o}{.}\PY{n}{unique}\PY{p}{(}\PY{p}{)}
\end{Verbatim}


\begin{Verbatim}[commandchars=\\\{\}]
{\color{outcolor}Out[{\color{outcolor}54}]:} array(['low', 'medium', 'high'], dtype=object)
\end{Verbatim}
            
    \begin{Verbatim}[commandchars=\\\{\}]
{\color{incolor}In [{\color{incolor}55}]:} \PY{c+c1}{\PYZsh{} count salary}
         \PY{n}{salary\PYZus{}grouped} \PY{o}{=} \PY{n}{pd}\PY{o}{.}\PY{n}{DataFrame}\PY{p}{(}\PY{n}{df}\PY{o}{.}\PY{n}{salary}\PY{o}{.}\PY{n}{groupby}\PY{p}{(}\PY{n}{by}\PY{o}{=}\PY{n}{df}\PY{o}{.}\PY{n}{salary}\PY{p}{)}\PY{o}{.}\PY{n}{count}\PY{p}{(}\PY{p}{)}\PY{p}{)}
         
         \PY{n+nb}{print}\PY{p}{(}\PY{n}{df}\PY{o}{.}\PY{n}{salary}\PY{o}{.}\PY{n}{groupby}\PY{p}{(}\PY{n}{by}\PY{o}{=}\PY{n}{df}\PY{o}{.}\PY{n}{salary}\PY{p}{)}\PY{o}{.}\PY{n}{count}\PY{p}{(}\PY{p}{)}\PY{p}{)}
         \PY{n+nb}{print}\PY{p}{(}\PY{l+s+s2}{\PYZdq{}}\PY{l+s+s2}{\PYZhy{}\PYZhy{}\PYZhy{}\PYZhy{}\PYZhy{}\PYZhy{}\PYZhy{}\PYZhy{}\PYZhy{}\PYZhy{}\PYZhy{}\PYZhy{}\PYZhy{}\PYZhy{}\PYZhy{}\PYZhy{}\PYZhy{}\PYZhy{}\PYZhy{}\PYZhy{}\PYZhy{}\PYZhy{}\PYZhy{}\PYZhy{}\PYZhy{}\PYZhy{}\PYZhy{}\PYZhy{}\PYZhy{}\PYZhy{}\PYZhy{}\PYZhy{}\PYZhy{}}\PY{l+s+s2}{\PYZdq{}}\PY{p}{)}
         \PY{n+nb}{print}\PY{p}{(}\PY{n}{df}\PY{o}{.}\PY{n}{salary}\PY{o}{.}\PY{n}{count}\PY{p}{(}\PY{p}{)}\PY{p}{)}
\end{Verbatim}


    \begin{Verbatim}[commandchars=\\\{\}]
salary
high      1237
low       7316
medium    6446
Name: salary, dtype: int64
---------------------------------
14999

    \end{Verbatim}

    \begin{Verbatim}[commandchars=\\\{\}]
{\color{incolor}In [{\color{incolor}85}]:} \PY{c+c1}{\PYZsh{} plotting the grouped unique values in the \PYZsq{}salary\PYZsq{} column}
         \PY{n}{salary\PYZus{}grouped}\PY{o}{.}\PY{n}{plot}\PY{p}{(}\PY{n}{kind}\PY{o}{=}\PY{l+s+s1}{\PYZsq{}}\PY{l+s+s1}{bar}\PY{l+s+s1}{\PYZsq{}}\PY{p}{,} \PY{n}{figsize}\PY{o}{=}\PY{p}{(}\PY{l+m+mi}{10}\PY{p}{,}\PY{l+m+mi}{6}\PY{p}{)}\PY{p}{)}
\end{Verbatim}


\begin{Verbatim}[commandchars=\\\{\}]
{\color{outcolor}Out[{\color{outcolor}85}]:} <matplotlib.axes.\_subplots.AxesSubplot at 0x2192188abe0>
\end{Verbatim}
            
    \begin{center}
    \adjustimage{max size={0.9\linewidth}{0.9\paperheight}}{output_27_1.png}
    \end{center}
    { \hspace*{\fill} \\}
    
    \paragraph{Dummy Encoding}\label{dummy-encoding}

Dummy coding refers to the process of coding a categorical variable into
dichotomous variables.

\href{http://pandas.pydata.org/pandas-docs/stable/generated/pandas.get_dummies.html}{Pandas
Dummy Encoding Documentation}

    \begin{Verbatim}[commandchars=\\\{\}]
{\color{incolor}In [{\color{incolor}61}]:} \PY{n}{salary\PYZus{}label} \PY{o}{=} \PY{n}{pd}\PY{o}{.}\PY{n}{get\PYZus{}dummies}\PY{p}{(}\PY{n}{df}\PY{o}{.}\PY{n}{salary}\PY{p}{,} \PY{n}{prefix}\PY{o}{=}\PY{l+s+s1}{\PYZsq{}}\PY{l+s+s1}{sal}\PY{l+s+s1}{\PYZsq{}}\PY{p}{)}
         \PY{n}{salary\PYZus{}label}\PY{o}{.}\PY{n}{head}\PY{p}{(}\PY{p}{)}
\end{Verbatim}


\begin{Verbatim}[commandchars=\\\{\}]
{\color{outcolor}Out[{\color{outcolor}61}]:}    sal\_high  sal\_low  sal\_medium
         0         0        1           0
         1         0        0           1
         2         0        0           1
         3         0        1           0
         4         0        1           0
\end{Verbatim}
            
    \subsubsection{Concatenating new columns to the original
DataFrame}\label{concatenating-new-columns-to-the-original-dataframe}

    \begin{Verbatim}[commandchars=\\\{\}]
{\color{incolor}In [{\color{incolor}64}]:} \PY{c+c1}{\PYZsh{} concatenating \PYZsq{}salary\PYZus{}label\PYZsq{} column}
         \PY{n}{df\PYZus{}new} \PY{o}{=} \PY{n}{pd}\PY{o}{.}\PY{n}{concat}\PY{p}{(}\PY{p}{[}\PY{n}{df}\PY{p}{,} \PY{n}{salary\PYZus{}label}\PY{p}{]}\PY{p}{,} \PY{n}{axis}\PY{o}{=}\PY{l+m+mi}{1}\PY{p}{,} \PY{n}{join}\PY{o}{=}\PY{l+s+s1}{\PYZsq{}}\PY{l+s+s1}{inner}\PY{l+s+s1}{\PYZsq{}}\PY{p}{)}
\end{Verbatim}


    \begin{Verbatim}[commandchars=\\\{\}]
{\color{incolor}In [{\color{incolor}65}]:} \PY{c+c1}{\PYZsh{} adding the \PYZsq{}sales\PYZus{}label\PYZsq{} column}
         \PY{n}{df\PYZus{}new}\PY{p}{[}\PY{l+s+s1}{\PYZsq{}}\PY{l+s+s1}{sales\PYZus{}encoded}\PY{l+s+s1}{\PYZsq{}}\PY{p}{]} \PY{o}{=} \PY{n}{sales\PYZus{}label}
         \PY{n}{df\PYZus{}new}\PY{o}{.}\PY{n}{head}\PY{p}{(}\PY{p}{)}
\end{Verbatim}


\begin{Verbatim}[commandchars=\\\{\}]
{\color{outcolor}Out[{\color{outcolor}65}]:}    satisfaction\_level  last\_evaluation  number\_project  average\_montly\_hours  \textbackslash{}
         0                0.38             0.53               2                   157   
         1                0.80             0.86               5                   262   
         2                0.11             0.88               7                   272   
         3                0.72             0.87               5                   223   
         4                0.37             0.52               2                   159   
         
            time\_spend\_company  Work\_accident  left  promotion\_last\_5years  sales  \textbackslash{}
         0                   3              0     1                      0  sales   
         1                   6              0     1                      0  sales   
         2                   4              0     1                      0  sales   
         3                   5              0     1                      0  sales   
         4                   3              0     1                      0  sales   
         
            salary  sal\_high  sal\_low  sal\_medium  sales\_encoded  
         0     low         0        1           0              7  
         1  medium         0        0           1              7  
         2  medium         0        0           1              7  
         3     low         0        1           0              7  
         4     low         0        1           0              7  
\end{Verbatim}
            
    \begin{Verbatim}[commandchars=\\\{\}]
{\color{incolor}In [{\color{incolor}66}]:} \PY{c+c1}{\PYZsh{} Removing \PYZsq{}sales\PYZsq{} and \PYZsq{}salary\PYZsq{} columns (character fields removed and substituted for model purpose}
         \PY{n}{df\PYZus{}new} \PY{o}{=} \PY{n}{df\PYZus{}new}\PY{o}{.}\PY{n}{drop}\PY{p}{(}\PY{p}{[}\PY{l+s+s1}{\PYZsq{}}\PY{l+s+s1}{sales}\PY{l+s+s1}{\PYZsq{}}\PY{p}{,}\PY{l+s+s1}{\PYZsq{}}\PY{l+s+s1}{salary}\PY{l+s+s1}{\PYZsq{}}\PY{p}{]}\PY{p}{,} \PY{n}{axis}\PY{o}{=}\PY{l+m+mi}{1}\PY{p}{)}
\end{Verbatim}


    \begin{Verbatim}[commandchars=\\\{\}]
{\color{incolor}In [{\color{incolor}67}]:} \PY{n}{df\PYZus{}new}\PY{o}{.}\PY{n}{head}\PY{p}{(}\PY{p}{)}
\end{Verbatim}


\begin{Verbatim}[commandchars=\\\{\}]
{\color{outcolor}Out[{\color{outcolor}67}]:}    satisfaction\_level  last\_evaluation  number\_project  average\_montly\_hours  \textbackslash{}
         0                0.38             0.53               2                   157   
         1                0.80             0.86               5                   262   
         2                0.11             0.88               7                   272   
         3                0.72             0.87               5                   223   
         4                0.37             0.52               2                   159   
         
            time\_spend\_company  Work\_accident  left  promotion\_last\_5years  sal\_high  \textbackslash{}
         0                   3              0     1                      0         0   
         1                   6              0     1                      0         0   
         2                   4              0     1                      0         0   
         3                   5              0     1                      0         0   
         4                   3              0     1                      0         0   
         
            sal\_low  sal\_medium  sales\_encoded  
         0        1           0              7  
         1        0           1              7  
         2        0           1              7  
         3        1           0              7  
         4        1           0              7  
\end{Verbatim}
            
    \begin{Verbatim}[commandchars=\\\{\}]
{\color{incolor}In [{\color{incolor}69}]:} \PY{c+c1}{\PYZsh{} As you can see, all columns are clearly numerical}
         \PY{n}{df\PYZus{}new}\PY{o}{.}\PY{n}{info}\PY{p}{(}\PY{p}{)}
\end{Verbatim}


    \begin{Verbatim}[commandchars=\\\{\}]
<class 'pandas.core.frame.DataFrame'>
RangeIndex: 14999 entries, 0 to 14998
Data columns (total 12 columns):
satisfaction\_level       14999 non-null float64
last\_evaluation          14999 non-null float64
number\_project           14999 non-null int64
average\_montly\_hours     14999 non-null int64
time\_spend\_company       14999 non-null int64
Work\_accident            14999 non-null int64
left                     14999 non-null int64
promotion\_last\_5years    14999 non-null int64
sal\_high                 14999 non-null uint8
sal\_low                  14999 non-null uint8
sal\_medium               14999 non-null uint8
sales\_encoded            14999 non-null int64
dtypes: float64(2), int64(7), uint8(3)
memory usage: 1.1 MB

    \end{Verbatim}

    \subsubsection{Separating FEATURES and
LABELS}\label{separating-features-and-labels}

    \begin{Verbatim}[commandchars=\\\{\}]
{\color{incolor}In [{\color{incolor}71}]:} \PY{c+c1}{\PYZsh{} X = features}
         \PY{n}{X} \PY{o}{=} \PY{n}{df\PYZus{}new}\PY{o}{.}\PY{n}{drop}\PY{p}{(}\PY{l+s+s1}{\PYZsq{}}\PY{l+s+s1}{left}\PY{l+s+s1}{\PYZsq{}}\PY{p}{,}\PY{n}{axis}\PY{o}{=}\PY{l+m+mi}{1}\PY{p}{)}
         \PY{n}{X}\PY{o}{.}\PY{n}{head}\PY{p}{(}\PY{p}{)}
\end{Verbatim}


\begin{Verbatim}[commandchars=\\\{\}]
{\color{outcolor}Out[{\color{outcolor}71}]:}    satisfaction\_level  last\_evaluation  number\_project  average\_montly\_hours  \textbackslash{}
         0                0.38             0.53               2                   157   
         1                0.80             0.86               5                   262   
         2                0.11             0.88               7                   272   
         3                0.72             0.87               5                   223   
         4                0.37             0.52               2                   159   
         
            time\_spend\_company  Work\_accident  promotion\_last\_5years  sal\_high  \textbackslash{}
         0                   3              0                      0         0   
         1                   6              0                      0         0   
         2                   4              0                      0         0   
         3                   5              0                      0         0   
         4                   3              0                      0         0   
         
            sal\_low  sal\_medium  sales\_encoded  
         0        1           0              7  
         1        0           1              7  
         2        0           1              7  
         3        1           0              7  
         4        1           0              7  
\end{Verbatim}
            
    \begin{Verbatim}[commandchars=\\\{\}]
{\color{incolor}In [{\color{incolor}72}]:} \PY{c+c1}{\PYZsh{} Y = labels}
         \PY{n}{Y} \PY{o}{=} \PY{n}{df\PYZus{}new}\PY{p}{[}\PY{l+s+s1}{\PYZsq{}}\PY{l+s+s1}{left}\PY{l+s+s1}{\PYZsq{}}\PY{p}{]}
         \PY{n}{Y}\PY{o}{.}\PY{n}{head}\PY{p}{(}\PY{p}{)}
\end{Verbatim}


\begin{Verbatim}[commandchars=\\\{\}]
{\color{outcolor}Out[{\color{outcolor}72}]:} 0    1
         1    1
         2    1
         3    1
         4    1
         Name: left, dtype: int64
\end{Verbatim}
            
    \section{DATA VISUALIZATION}\label{data-visualization}

    \begin{Verbatim}[commandchars=\\\{\}]
{\color{incolor}In [{\color{incolor}73}]:} \PY{c+c1}{\PYZsh{}\PYZsh{} Visualizing the whole dataset}
         \PY{n}{sns}\PY{o}{.}\PY{n}{pairplot}\PY{p}{(}\PY{n}{df\PYZus{}new}\PY{p}{,} \PY{n}{hue}\PY{o}{=}\PY{l+s+s1}{\PYZsq{}}\PY{l+s+s1}{left}\PY{l+s+s1}{\PYZsq{}}\PY{p}{)}
\end{Verbatim}


\begin{Verbatim}[commandchars=\\\{\}]
{\color{outcolor}Out[{\color{outcolor}73}]:} <seaborn.axisgrid.PairGrid at 0x2191275c160>
\end{Verbatim}
            
    \begin{center}
    \adjustimage{max size={0.9\linewidth}{0.9\paperheight}}{output_40_1.png}
    \end{center}
    { \hspace*{\fill} \\}
    
    \begin{Verbatim}[commandchars=\\\{\}]
{\color{incolor}In [{\color{incolor}94}]:} \PY{c+c1}{\PYZsh{} Distribution of time spent in the company}
         \PY{n}{pd}\PY{o}{.}\PY{n}{DataFrame}\PY{p}{(}\PY{n}{df\PYZus{}new}\PY{o}{.}\PY{n}{time\PYZus{}spend\PYZus{}company}\PY{o}{.}\PY{n}{groupby}\PY{p}{(}\PY{n}{by}\PY{o}{=}\PY{n}{df\PYZus{}new}\PY{o}{.}\PY{n}{time\PYZus{}spend\PYZus{}company}\PY{p}{)}\PY{o}{.}\PY{n}{count}\PY{p}{(}\PY{p}{)}\PY{p}{)}\PY{o}{.}\PY{n}{plot}\PY{p}{(}\PY{n}{kind}\PY{o}{=}\PY{l+s+s1}{\PYZsq{}}\PY{l+s+s1}{bar}\PY{l+s+s1}{\PYZsq{}}\PY{p}{,} \PY{n}{figsize}\PY{o}{=}\PY{p}{(}\PY{l+m+mi}{13}\PY{p}{,}\PY{l+m+mi}{6}\PY{p}{)}\PY{p}{,} \PY{n}{title}\PY{o}{=}\PY{l+s+s1}{\PYZsq{}}\PY{l+s+s1}{Years spent in the Company}\PY{l+s+s1}{\PYZsq{}}\PY{p}{)}
\end{Verbatim}


\begin{Verbatim}[commandchars=\\\{\}]
{\color{outcolor}Out[{\color{outcolor}94}]:} <matplotlib.axes.\_subplots.AxesSubplot at 0x21920aa1ef0>
\end{Verbatim}
            
    \begin{center}
    \adjustimage{max size={0.9\linewidth}{0.9\paperheight}}{output_41_1.png}
    \end{center}
    { \hspace*{\fill} \\}
    
    \begin{Verbatim}[commandchars=\\\{\}]
{\color{incolor}In [{\color{incolor}96}]:} \PY{c+c1}{\PYZsh{} plotting the grouped unique values in the \PYZsq{}sales\PYZsq{} column}
         \PY{n}{sales\PYZus{}grouped}\PY{o}{.}\PY{n}{plot}\PY{p}{(}\PY{n}{kind}\PY{o}{=}\PY{l+s+s1}{\PYZsq{}}\PY{l+s+s1}{bar}\PY{l+s+s1}{\PYZsq{}}\PY{p}{,} \PY{n}{figsize}\PY{o}{=}\PY{p}{(}\PY{l+m+mi}{15}\PY{p}{,}\PY{l+m+mi}{6}\PY{p}{)}\PY{p}{,} \PY{n}{title}\PY{o}{=}\PY{l+s+s2}{\PYZdq{}}\PY{l+s+s2}{Count plot for all Departments}\PY{l+s+s2}{\PYZdq{}}\PY{p}{)}
\end{Verbatim}


\begin{Verbatim}[commandchars=\\\{\}]
{\color{outcolor}Out[{\color{outcolor}96}]:} <matplotlib.axes.\_subplots.AxesSubplot at 0x219213f8d68>
\end{Verbatim}
            
    \begin{center}
    \adjustimage{max size={0.9\linewidth}{0.9\paperheight}}{output_42_1.png}
    \end{center}
    { \hspace*{\fill} \\}
    
    \begin{Verbatim}[commandchars=\\\{\}]
{\color{incolor}In [{\color{incolor}99}]:} \PY{c+c1}{\PYZsh{} plotting the grouped unique values in the \PYZsq{}salary\PYZsq{} column}
         \PY{n}{salary\PYZus{}grouped}\PY{o}{.}\PY{n}{plot}\PY{p}{(}\PY{n}{kind}\PY{o}{=}\PY{l+s+s1}{\PYZsq{}}\PY{l+s+s1}{bar}\PY{l+s+s1}{\PYZsq{}}\PY{p}{,} \PY{n}{figsize}\PY{o}{=}\PY{p}{(}\PY{l+m+mi}{10}\PY{p}{,}\PY{l+m+mi}{6}\PY{p}{)}\PY{p}{,} \PY{n}{title}\PY{o}{=}\PY{l+s+s1}{\PYZsq{}}\PY{l+s+s1}{Frequency of all }\PY{l+s+s1}{\PYZdq{}}\PY{l+s+s1}{salary}\PY{l+s+s1}{\PYZdq{}}\PY{l+s+s1}{ categories}\PY{l+s+s1}{\PYZsq{}}\PY{p}{)}
\end{Verbatim}


\begin{Verbatim}[commandchars=\\\{\}]
{\color{outcolor}Out[{\color{outcolor}99}]:} <matplotlib.axes.\_subplots.AxesSubplot at 0x21921a782e8>
\end{Verbatim}
            
    \begin{center}
    \adjustimage{max size={0.9\linewidth}{0.9\paperheight}}{output_43_1.png}
    \end{center}
    { \hspace*{\fill} \\}
    
    \section{DATA NORMALIZATION}\label{data-normalization}

    \subsection{Feature Scaling}\label{feature-scaling}

\begin{itemize}
\tightlist
\item
  Feature scaling is a method used to standardize the range of
  independent variables or features of data.
\item
  In data processing, it is also known as data normalization and is
  generally performed during the data preprocessing step.
\end{itemize}

Mean Normalization \#\#\#
\[x' = \frac{x - \text{mean}(x)}{\text{max}(x)-\text{min}(x)}\]

    \subsubsection{Standard Scaler}\label{standard-scaler}

\begin{enumerate}
\def\labelenumi{\arabic{enumi}.}
\item
  Standardize features by removing the mean and scaling to unit variance
\item
  Centering and scaling happen independently on each feature by
  computing the relevant statistics on the samples in the training set.
\item
  Mean and standard deviation are then stored to be used on later data
  using the transform method.
\end{enumerate}

Standardization of a dataset is a common requirement for many machine
learning estimators: they might behave badly if the individual feature
do not more or less look like standard normally distributed data (e.g.
Gaussian with 0 mean and unit variance).

\subsection{\texorpdfstring{\[x' = \frac{x - \bar{x}}{\sigma}\]}{x' = \textbackslash{}frac\{x - \textbackslash{}bar\{x\}\}\{\textbackslash{}sigma\}}}\label{x-fracx---barxsigma}

where \({\sigma}\) is standard deviation and \(\bar{x}\) is mean.

    \begin{Verbatim}[commandchars=\\\{\}]
{\color{incolor}In [{\color{incolor}102}]:} \PY{c+c1}{\PYZsh{} Scaling the data (for faster training)}
          \PY{k+kn}{from} \PY{n+nn}{sklearn}\PY{n+nn}{.}\PY{n+nn}{preprocessing} \PY{k}{import} \PY{n}{StandardScaler}
\end{Verbatim}


    \begin{Verbatim}[commandchars=\\\{\}]
{\color{incolor}In [{\color{incolor}104}]:} \PY{c+c1}{\PYZsh{} initializing StandardScaler}
          \PY{n}{scalar} \PY{o}{=} \PY{n}{StandardScaler}\PY{p}{(}\PY{p}{)}
\end{Verbatim}


    \begin{Verbatim}[commandchars=\\\{\}]
{\color{incolor}In [{\color{incolor}105}]:} \PY{c+c1}{\PYZsh{} transforming the data}
          \PY{n}{X\PYZus{}normalized} \PY{o}{=} \PY{n}{scalar}\PY{o}{.}\PY{n}{fit\PYZus{}transform}\PY{p}{(}\PY{n}{X}\PY{p}{)}
          \PY{n}{X\PYZus{}normalized}
\end{Verbatim}


\begin{Verbatim}[commandchars=\\\{\}]
{\color{outcolor}Out[{\color{outcolor}105}]:} array([[-0.93649469, -1.08727529, -1.46286291, {\ldots},  1.02477511,
                  -0.8681323 ,  0.39372503],
                 [ 0.75281433,  0.84070693,  0.97111292, {\ldots}, -0.97582386,
                   1.15189816,  0.39372503],
                 [-2.02247906,  0.95755433,  2.59376348, {\ldots}, -0.97582386,
                   1.15189816,  0.39372503],
                 {\ldots}, 
                 [-0.97671633, -1.08727529, -1.46286291, {\ldots},  1.02477511,
                  -0.8681323 ,  0.74231612],
                 [-2.02247906,  1.42494396,  1.7824382 , {\ldots},  1.02477511,
                  -0.8681323 ,  0.74231612],
                 [-0.97671633, -1.14569899, -1.46286291, {\ldots},  1.02477511,
                  -0.8681323 ,  0.74231612]])
\end{Verbatim}
            
    \section{Training and
Cross-Validation}\label{training-and-cross-validation}

    \begin{Verbatim}[commandchars=\\\{\}]
{\color{incolor}In [{\color{incolor}106}]:} \PY{c+c1}{\PYZsh{} train\PYZhy{}test split}
          \PY{c+c1}{\PYZsh{} ideal test size is 30\PYZpc{}}
          \PY{k+kn}{from} \PY{n+nn}{sklearn}\PY{n+nn}{.}\PY{n+nn}{model\PYZus{}selection} \PY{k}{import} \PY{n}{train\PYZus{}test\PYZus{}split}
          
          \PY{n}{x\PYZus{}train}\PY{p}{,} \PY{n}{x\PYZus{}test}\PY{p}{,} \PY{n}{y\PYZus{}train}\PY{p}{,} \PY{n}{y\PYZus{}test} \PY{o}{=} \PY{n}{train\PYZus{}test\PYZus{}split}\PY{p}{(}\PY{n}{X}\PY{p}{,} \PY{n}{Y}\PY{p}{,} \PY{n}{test\PYZus{}size}\PY{o}{=}\PY{l+m+mf}{0.3}\PY{p}{,} \PY{n}{random\PYZus{}state}\PY{o}{=}\PY{l+m+mi}{101}\PY{p}{)}
\end{Verbatim}


    \begin{Verbatim}[commandchars=\\\{\}]
{\color{incolor}In [{\color{incolor}107}]:} \PY{c+c1}{\PYZsh{} shape of the training features}
          \PY{n}{x\PYZus{}train}\PY{o}{.}\PY{n}{shape}
\end{Verbatim}


\begin{Verbatim}[commandchars=\\\{\}]
{\color{outcolor}Out[{\color{outcolor}107}]:} (10499, 11)
\end{Verbatim}
            
    \begin{Verbatim}[commandchars=\\\{\}]
{\color{incolor}In [{\color{incolor}108}]:} \PY{c+c1}{\PYZsh{} shape of the training labels}
          \PY{n}{y\PYZus{}train}\PY{o}{.}\PY{n}{shape}
\end{Verbatim}


\begin{Verbatim}[commandchars=\\\{\}]
{\color{outcolor}Out[{\color{outcolor}108}]:} (10499,)
\end{Verbatim}
            
    \begin{Verbatim}[commandchars=\\\{\}]
{\color{incolor}In [{\color{incolor}109}]:} \PY{c+c1}{\PYZsh{} shape of the testing features}
          \PY{n}{x\PYZus{}test}\PY{o}{.}\PY{n}{shape}
\end{Verbatim}


\begin{Verbatim}[commandchars=\\\{\}]
{\color{outcolor}Out[{\color{outcolor}109}]:} (4500, 11)
\end{Verbatim}
            
    \begin{Verbatim}[commandchars=\\\{\}]
{\color{incolor}In [{\color{incolor}110}]:} \PY{c+c1}{\PYZsh{} shape of the testing labels}
          \PY{n}{y\PYZus{}test}\PY{o}{.}\PY{n}{shape}
\end{Verbatim}


\begin{Verbatim}[commandchars=\\\{\}]
{\color{outcolor}Out[{\color{outcolor}110}]:} (4500,)
\end{Verbatim}
            
    \begin{Verbatim}[commandchars=\\\{\}]
{\color{incolor}In [{\color{incolor}111}]:} \PY{c+c1}{\PYZsh{} importing models}
          \PY{k+kn}{from} \PY{n+nn}{sklearn}\PY{n+nn}{.}\PY{n+nn}{linear\PYZus{}model} \PY{k}{import} \PY{n}{LogisticRegression}
          \PY{k+kn}{from} \PY{n+nn}{sklearn}\PY{n+nn}{.}\PY{n+nn}{neighbors} \PY{k}{import} \PY{n}{KNeighborsClassifier}
          \PY{k+kn}{from} \PY{n+nn}{sklearn}\PY{n+nn}{.}\PY{n+nn}{svm} \PY{k}{import} \PY{n}{SVC}
          \PY{k+kn}{from} \PY{n+nn}{sklearn}\PY{n+nn}{.}\PY{n+nn}{tree} \PY{k}{import} \PY{n}{DecisionTreeClassifier}
          \PY{k+kn}{from} \PY{n+nn}{sklearn}\PY{n+nn}{.}\PY{n+nn}{ensemble} \PY{k}{import} \PY{n}{RandomForestClassifier}\PY{p}{,} \PY{n}{AdaBoostClassifier}
          \PY{k+kn}{from} \PY{n+nn}{sklearn}\PY{n+nn}{.}\PY{n+nn}{naive\PYZus{}bayes} \PY{k}{import} \PY{n}{GaussianNB}
          \PY{k+kn}{from} \PY{n+nn}{sklearn}\PY{n+nn}{.}\PY{n+nn}{discriminant\PYZus{}analysis} \PY{k}{import} \PY{n}{QuadraticDiscriminantAnalysis}
          \PY{k+kn}{from} \PY{n+nn}{sklearn}\PY{n+nn}{.}\PY{n+nn}{discriminant\PYZus{}analysis} \PY{k}{import} \PY{n}{LinearDiscriminantAnalysis}
\end{Verbatim}


    \begin{Verbatim}[commandchars=\\\{\}]
{\color{incolor}In [{\color{incolor}112}]:} \PY{c+c1}{\PYZsh{} initializing all the models}
          \PY{n}{models} \PY{o}{=} \PY{p}{[}\PY{p}{]}
          \PY{n}{models}\PY{o}{.}\PY{n}{append}\PY{p}{(}\PY{p}{[}\PY{l+s+s1}{\PYZsq{}}\PY{l+s+s1}{LR}\PY{l+s+s1}{\PYZsq{}}\PY{p}{,}\PY{n}{LogisticRegression}\PY{p}{(}\PY{p}{)}\PY{p}{]}\PY{p}{)}
          \PY{n}{models}\PY{o}{.}\PY{n}{append}\PY{p}{(}\PY{p}{[}\PY{l+s+s1}{\PYZsq{}}\PY{l+s+s1}{KNN}\PY{l+s+s1}{\PYZsq{}}\PY{p}{,}\PY{n}{KNeighborsClassifier}\PY{p}{(}\PY{p}{)}\PY{p}{]}\PY{p}{)}
          \PY{n}{models}\PY{o}{.}\PY{n}{append}\PY{p}{(}\PY{p}{[}\PY{l+s+s1}{\PYZsq{}}\PY{l+s+s1}{SVM}\PY{l+s+s1}{\PYZsq{}}\PY{p}{,}\PY{n}{SVC}\PY{p}{(}\PY{p}{)}\PY{p}{]}\PY{p}{)}
          \PY{n}{models}\PY{o}{.}\PY{n}{append}\PY{p}{(}\PY{p}{[}\PY{l+s+s1}{\PYZsq{}}\PY{l+s+s1}{DT}\PY{l+s+s1}{\PYZsq{}}\PY{p}{,}\PY{n}{DecisionTreeClassifier}\PY{p}{(}\PY{p}{)}\PY{p}{]}\PY{p}{)}
          \PY{n}{models}\PY{o}{.}\PY{n}{append}\PY{p}{(}\PY{p}{[}\PY{l+s+s1}{\PYZsq{}}\PY{l+s+s1}{RF}\PY{l+s+s1}{\PYZsq{}}\PY{p}{,}\PY{n}{RandomForestClassifier}\PY{p}{(}\PY{p}{)}\PY{p}{]}\PY{p}{)}
          \PY{n}{models}\PY{o}{.}\PY{n}{append}\PY{p}{(}\PY{p}{[}\PY{l+s+s1}{\PYZsq{}}\PY{l+s+s1}{AB}\PY{l+s+s1}{\PYZsq{}}\PY{p}{,}\PY{n}{AdaBoostClassifier}\PY{p}{(}\PY{p}{)}\PY{p}{]}\PY{p}{)}
          \PY{n}{models}\PY{o}{.}\PY{n}{append}\PY{p}{(}\PY{p}{[}\PY{l+s+s1}{\PYZsq{}}\PY{l+s+s1}{GNB}\PY{l+s+s1}{\PYZsq{}}\PY{p}{,}\PY{n}{GaussianNB}\PY{p}{(}\PY{p}{)}\PY{p}{]}\PY{p}{)}
          \PY{n}{models}\PY{o}{.}\PY{n}{append}\PY{p}{(}\PY{p}{[}\PY{l+s+s1}{\PYZsq{}}\PY{l+s+s1}{QDA}\PY{l+s+s1}{\PYZsq{}}\PY{p}{,}\PY{n}{QuadraticDiscriminantAnalysis}\PY{p}{(}\PY{p}{)}\PY{p}{]}\PY{p}{)}
          \PY{n}{models}\PY{o}{.}\PY{n}{append}\PY{p}{(}\PY{p}{[}\PY{l+s+s1}{\PYZsq{}}\PY{l+s+s1}{LDA}\PY{l+s+s1}{\PYZsq{}}\PY{p}{,}\PY{n}{LinearDiscriminantAnalysis}\PY{p}{(}\PY{p}{)}\PY{p}{]}\PY{p}{)}
\end{Verbatim}


    \begin{Verbatim}[commandchars=\\\{\}]
{\color{incolor}In [{\color{incolor}113}]:} \PY{n+nb}{len}\PY{p}{(}\PY{n}{models}\PY{p}{)}
\end{Verbatim}


\begin{Verbatim}[commandchars=\\\{\}]
{\color{outcolor}Out[{\color{outcolor}113}]:} 9
\end{Verbatim}
            
    \begin{Verbatim}[commandchars=\\\{\}]
{\color{incolor}In [{\color{incolor}114}]:} \PY{k+kn}{from} \PY{n+nn}{sklearn}\PY{n+nn}{.}\PY{n+nn}{cross\PYZus{}validation} \PY{k}{import} \PY{n}{KFold}
          \PY{k+kn}{from} \PY{n+nn}{sklearn}\PY{n+nn}{.}\PY{n+nn}{model\PYZus{}selection} \PY{k}{import} \PY{n}{cross\PYZus{}val\PYZus{}score}
          \PY{n}{kfold} \PY{o}{=} \PY{n}{KFold}\PY{p}{(}\PY{l+m+mi}{1000}\PY{p}{,} \PY{n}{n\PYZus{}folds}\PY{o}{=}\PY{l+m+mi}{10}\PY{p}{)}
\end{Verbatim}


    \begin{Verbatim}[commandchars=\\\{\}]
C:\textbackslash{}Users\textbackslash{}shashwat\textbackslash{}Anaconda3\textbackslash{}lib\textbackslash{}site-packages\textbackslash{}sklearn\textbackslash{}cross\_validation.py:41: DeprecationWarning: This module was deprecated in version 0.18 in favor of the model\_selection module into which all the refactored classes and functions are moved. Also note that the interface of the new CV iterators are different from that of this module. This module will be removed in 0.20.
  "This module will be removed in 0.20.", DeprecationWarning)

    \end{Verbatim}

    \begin{Verbatim}[commandchars=\\\{\}]
{\color{incolor}In [{\color{incolor}116}]:} \PY{n}{results} \PY{o}{=} \PY{p}{[}\PY{p}{]}
          \PY{n}{names} \PY{o}{=} \PY{p}{[}\PY{p}{]}
          \PY{k}{for} \PY{n}{name}\PY{p}{,} \PY{n}{model} \PY{o+ow}{in} \PY{n}{models}\PY{p}{:}
              \PY{n+nb}{print}\PY{p}{(}\PY{l+s+s1}{\PYZsq{}}\PY{l+s+s1}{training...}\PY{l+s+s1}{\PYZsq{}} \PY{o}{+} \PY{n}{name}\PY{p}{)}
              \PY{n}{scores} \PY{o}{=} \PY{n}{cross\PYZus{}val\PYZus{}score}\PY{p}{(}\PY{n}{model}\PY{p}{,} \PY{n}{x\PYZus{}train}\PY{p}{,} \PY{n}{y\PYZus{}train}\PY{p}{,} \PY{n}{cv}\PY{o}{=}\PY{n}{kfold}\PY{p}{,} \PY{n}{scoring}\PY{o}{=}\PY{l+s+s1}{\PYZsq{}}\PY{l+s+s1}{accuracy}\PY{l+s+s1}{\PYZsq{}}\PY{p}{)}
              \PY{n}{names}\PY{o}{.}\PY{n}{append}\PY{p}{(}\PY{n}{name}\PY{p}{)}
              \PY{n}{results}\PY{o}{.}\PY{n}{append}\PY{p}{(}\PY{p}{[} \PY{n}{np}\PY{o}{.}\PY{n}{mean}\PY{p}{(}\PY{n}{scores}\PY{p}{)}\PY{p}{,} \PY{n}{np}\PY{o}{.}\PY{n}{std}\PY{p}{(}\PY{n}{scores}\PY{p}{)}\PY{p}{]}\PY{p}{)}
              \PY{n+nb}{print}\PY{p}{(}\PY{n}{name} \PY{o}{+} \PY{l+s+s1}{\PYZsq{}}\PY{l+s+s1}{ accuracy(mean):}\PY{l+s+s1}{\PYZsq{}} \PY{o}{+} \PY{n+nb}{str}\PY{p}{(}\PY{n}{np}\PY{o}{.}\PY{n}{mean}\PY{p}{(}\PY{n}{scores}\PY{p}{)}\PY{p}{)}\PY{p}{)}
              \PY{n+nb}{print}\PY{p}{(}\PY{n}{name} \PY{o}{+} \PY{l+s+s1}{\PYZsq{}}\PY{l+s+s1}{ accuracy(std):}\PY{l+s+s1}{\PYZsq{}} \PY{o}{+} \PY{n+nb}{str}\PY{p}{(}\PY{n}{np}\PY{o}{.}\PY{n}{std}\PY{p}{(}\PY{n}{scores}\PY{p}{)}\PY{p}{)}\PY{p}{)}
              \PY{n+nb}{print}\PY{p}{(}\PY{l+s+s1}{\PYZsq{}}\PY{l+s+s1}{.......................................}\PY{l+s+s1}{\PYZsq{}}\PY{p}{)}
\end{Verbatim}


    \begin{Verbatim}[commandchars=\\\{\}]
training{\ldots}LR
LR accuracy(mean):0.781
LR accuracy(std):0.029478805946
{\ldots}
training{\ldots}KNN
KNN accuracy(mean):0.879
KNN accuracy(std):0.0406078810085
{\ldots}
training{\ldots}SVM
SVM accuracy(mean):0.904
SVM accuracy(std):0.0429418211072
{\ldots}
training{\ldots}DT
DT accuracy(mean):0.95
DT accuracy(std):0.0209761769634
{\ldots}
training{\ldots}RF
RF accuracy(mean):0.972
RF accuracy(std):0.0188679622641
{\ldots}
training{\ldots}AB
AB accuracy(mean):0.948
AB accuracy(std):0.0153622914957
{\ldots}
training{\ldots}GNB
GNB accuracy(mean):0.688
GNB accuracy(std):0.0975499871861
{\ldots}
training{\ldots}QDA
QDA accuracy(mean):0.642
QDA accuracy(std):0.180875647891
{\ldots}
training{\ldots}LDA
LDA accuracy(mean):0.781
LDA accuracy(std):0.0284429253067
{\ldots}

    \end{Verbatim}

    \begin{Verbatim}[commandchars=\\\{\}]
C:\textbackslash{}Users\textbackslash{}shashwat\textbackslash{}Anaconda3\textbackslash{}lib\textbackslash{}site-packages\textbackslash{}sklearn\textbackslash{}discriminant\_analysis.py:682: UserWarning: Variables are collinear
  warnings.warn("Variables are collinear")
C:\textbackslash{}Users\textbackslash{}shashwat\textbackslash{}Anaconda3\textbackslash{}lib\textbackslash{}site-packages\textbackslash{}sklearn\textbackslash{}discriminant\_analysis.py:682: UserWarning: Variables are collinear
  warnings.warn("Variables are collinear")
C:\textbackslash{}Users\textbackslash{}shashwat\textbackslash{}Anaconda3\textbackslash{}lib\textbackslash{}site-packages\textbackslash{}sklearn\textbackslash{}discriminant\_analysis.py:682: UserWarning: Variables are collinear
  warnings.warn("Variables are collinear")
C:\textbackslash{}Users\textbackslash{}shashwat\textbackslash{}Anaconda3\textbackslash{}lib\textbackslash{}site-packages\textbackslash{}sklearn\textbackslash{}discriminant\_analysis.py:682: UserWarning: Variables are collinear
  warnings.warn("Variables are collinear")
C:\textbackslash{}Users\textbackslash{}shashwat\textbackslash{}Anaconda3\textbackslash{}lib\textbackslash{}site-packages\textbackslash{}sklearn\textbackslash{}discriminant\_analysis.py:682: UserWarning: Variables are collinear
  warnings.warn("Variables are collinear")
C:\textbackslash{}Users\textbackslash{}shashwat\textbackslash{}Anaconda3\textbackslash{}lib\textbackslash{}site-packages\textbackslash{}sklearn\textbackslash{}discriminant\_analysis.py:682: UserWarning: Variables are collinear
  warnings.warn("Variables are collinear")
C:\textbackslash{}Users\textbackslash{}shashwat\textbackslash{}Anaconda3\textbackslash{}lib\textbackslash{}site-packages\textbackslash{}sklearn\textbackslash{}discriminant\_analysis.py:682: UserWarning: Variables are collinear
  warnings.warn("Variables are collinear")
C:\textbackslash{}Users\textbackslash{}shashwat\textbackslash{}Anaconda3\textbackslash{}lib\textbackslash{}site-packages\textbackslash{}sklearn\textbackslash{}discriminant\_analysis.py:682: UserWarning: Variables are collinear
  warnings.warn("Variables are collinear")
C:\textbackslash{}Users\textbackslash{}shashwat\textbackslash{}Anaconda3\textbackslash{}lib\textbackslash{}site-packages\textbackslash{}sklearn\textbackslash{}discriminant\_analysis.py:682: UserWarning: Variables are collinear
  warnings.warn("Variables are collinear")
C:\textbackslash{}Users\textbackslash{}shashwat\textbackslash{}Anaconda3\textbackslash{}lib\textbackslash{}site-packages\textbackslash{}sklearn\textbackslash{}discriminant\_analysis.py:682: UserWarning: Variables are collinear
  warnings.warn("Variables are collinear")
C:\textbackslash{}Users\textbackslash{}shashwat\textbackslash{}Anaconda3\textbackslash{}lib\textbackslash{}site-packages\textbackslash{}sklearn\textbackslash{}discriminant\_analysis.py:388: UserWarning: Variables are collinear.
  warnings.warn("Variables are collinear.")
C:\textbackslash{}Users\textbackslash{}shashwat\textbackslash{}Anaconda3\textbackslash{}lib\textbackslash{}site-packages\textbackslash{}sklearn\textbackslash{}discriminant\_analysis.py:388: UserWarning: Variables are collinear.
  warnings.warn("Variables are collinear.")
C:\textbackslash{}Users\textbackslash{}shashwat\textbackslash{}Anaconda3\textbackslash{}lib\textbackslash{}site-packages\textbackslash{}sklearn\textbackslash{}discriminant\_analysis.py:388: UserWarning: Variables are collinear.
  warnings.warn("Variables are collinear.")
C:\textbackslash{}Users\textbackslash{}shashwat\textbackslash{}Anaconda3\textbackslash{}lib\textbackslash{}site-packages\textbackslash{}sklearn\textbackslash{}discriminant\_analysis.py:388: UserWarning: Variables are collinear.
  warnings.warn("Variables are collinear.")
C:\textbackslash{}Users\textbackslash{}shashwat\textbackslash{}Anaconda3\textbackslash{}lib\textbackslash{}site-packages\textbackslash{}sklearn\textbackslash{}discriminant\_analysis.py:388: UserWarning: Variables are collinear.
  warnings.warn("Variables are collinear.")
C:\textbackslash{}Users\textbackslash{}shashwat\textbackslash{}Anaconda3\textbackslash{}lib\textbackslash{}site-packages\textbackslash{}sklearn\textbackslash{}discriminant\_analysis.py:388: UserWarning: Variables are collinear.
  warnings.warn("Variables are collinear.")
C:\textbackslash{}Users\textbackslash{}shashwat\textbackslash{}Anaconda3\textbackslash{}lib\textbackslash{}site-packages\textbackslash{}sklearn\textbackslash{}discriminant\_analysis.py:388: UserWarning: Variables are collinear.
  warnings.warn("Variables are collinear.")
C:\textbackslash{}Users\textbackslash{}shashwat\textbackslash{}Anaconda3\textbackslash{}lib\textbackslash{}site-packages\textbackslash{}sklearn\textbackslash{}discriminant\_analysis.py:388: UserWarning: Variables are collinear.
  warnings.warn("Variables are collinear.")
C:\textbackslash{}Users\textbackslash{}shashwat\textbackslash{}Anaconda3\textbackslash{}lib\textbackslash{}site-packages\textbackslash{}sklearn\textbackslash{}discriminant\_analysis.py:388: UserWarning: Variables are collinear.
  warnings.warn("Variables are collinear.")
C:\textbackslash{}Users\textbackslash{}shashwat\textbackslash{}Anaconda3\textbackslash{}lib\textbackslash{}site-packages\textbackslash{}sklearn\textbackslash{}discriminant\_analysis.py:388: UserWarning: Variables are collinear.
  warnings.warn("Variables are collinear.")

    \end{Verbatim}

    \subsection{Evaluation}\label{evaluation}

    \begin{Verbatim}[commandchars=\\\{\}]
{\color{incolor}In [{\color{incolor}117}]:} \PY{n}{results}
\end{Verbatim}


\begin{Verbatim}[commandchars=\\\{\}]
{\color{outcolor}Out[{\color{outcolor}117}]:} [[0.78100000000000003, 0.02947880594596736],
           [0.87899999999999989, 0.040607881008493926],
           [0.90400000000000014, 0.042941821107167787],
           [0.94999999999999996, 0.020976176963403009],
           [0.97200000000000009, 0.018867962264113199],
           [0.94800000000000006, 0.015362291495737196],
           [0.68799999999999994, 0.097549987186057591],
           [0.64200000000000002, 0.18087564789103039],
           [0.78099999999999992, 0.028442925306655785]]
\end{Verbatim}
            
    \begin{Verbatim}[commandchars=\\\{\}]
{\color{incolor}In [{\color{incolor}118}]:} \PY{n}{fig} \PY{o}{=} \PY{n}{plt}\PY{o}{.}\PY{n}{figure}\PY{p}{(}\PY{n}{figsize}\PY{o}{=}\PY{p}{(}\PY{l+m+mi}{15}\PY{p}{,}\PY{l+m+mi}{6}\PY{p}{)}\PY{p}{)}
          \PY{n}{fig}\PY{o}{.}\PY{n}{suptitle}\PY{p}{(}\PY{l+s+s1}{\PYZsq{}}\PY{l+s+s1}{Algorithm Comparison}\PY{l+s+s1}{\PYZsq{}}\PY{p}{)}
          \PY{n}{ax} \PY{o}{=} \PY{n}{fig}\PY{o}{.}\PY{n}{add\PYZus{}subplot}\PY{p}{(}\PY{l+m+mi}{111}\PY{p}{)}
          \PY{n}{plt}\PY{o}{.}\PY{n}{boxplot}\PY{p}{(}\PY{n}{results}\PY{p}{)}
          \PY{n}{ax}\PY{o}{.}\PY{n}{set\PYZus{}xticklabels}\PY{p}{(}\PY{n}{names}\PY{p}{)}
          \PY{n}{plt}\PY{o}{.}\PY{n}{show}\PY{p}{(}\PY{p}{)}
\end{Verbatim}


    \begin{center}
    \adjustimage{max size={0.9\linewidth}{0.9\paperheight}}{output_63_0.png}
    \end{center}
    { \hspace*{\fill} \\}
    
    \begin{Verbatim}[commandchars=\\\{\}]
{\color{incolor}In [{\color{incolor}119}]:} \PY{n}{random\PYZus{}forest} \PY{o}{=} \PY{n}{RandomForestClassifier}\PY{p}{(}\PY{p}{)}
\end{Verbatim}


    \begin{Verbatim}[commandchars=\\\{\}]
{\color{incolor}In [{\color{incolor}120}]:} \PY{n}{random\PYZus{}forest}\PY{o}{.}\PY{n}{fit}\PY{p}{(}\PY{n}{x\PYZus{}train}\PY{p}{,} \PY{n}{y\PYZus{}train}\PY{p}{)}
\end{Verbatim}


\begin{Verbatim}[commandchars=\\\{\}]
{\color{outcolor}Out[{\color{outcolor}120}]:} RandomForestClassifier(bootstrap=True, class\_weight=None, criterion='gini',
                      max\_depth=None, max\_features='auto', max\_leaf\_nodes=None,
                      min\_impurity\_decrease=0.0, min\_impurity\_split=None,
                      min\_samples\_leaf=1, min\_samples\_split=2,
                      min\_weight\_fraction\_leaf=0.0, n\_estimators=10, n\_jobs=1,
                      oob\_score=False, random\_state=None, verbose=0,
                      warm\_start=False)
\end{Verbatim}
            
    \begin{Verbatim}[commandchars=\\\{\}]
{\color{incolor}In [{\color{incolor}121}]:} \PY{k+kn}{from} \PY{n+nn}{sklearn}\PY{n+nn}{.}\PY{n+nn}{metrics} \PY{k}{import} \PY{n}{classification\PYZus{}report}\PY{p}{,}\PY{n}{confusion\PYZus{}matrix}
\end{Verbatim}


    \begin{Verbatim}[commandchars=\\\{\}]
{\color{incolor}In [{\color{incolor}122}]:} \PY{n}{predict} \PY{o}{=} \PY{n}{random\PYZus{}forest}\PY{o}{.}\PY{n}{predict}\PY{p}{(}\PY{n}{x\PYZus{}test}\PY{p}{)}
\end{Verbatim}


    \begin{Verbatim}[commandchars=\\\{\}]
{\color{incolor}In [{\color{incolor}123}]:} \PY{n+nb}{print}\PY{p}{(}\PY{n}{classification\PYZus{}report}\PY{p}{(}\PY{n}{y\PYZus{}test}\PY{p}{,} \PY{n}{predict}\PY{p}{)}\PY{p}{)}
\end{Verbatim}


    \begin{Verbatim}[commandchars=\\\{\}]
             precision    recall  f1-score   support

          0       0.99      1.00      0.99      3431
          1       0.99      0.96      0.98      1069

avg / total       0.99      0.99      0.99      4500


    \end{Verbatim}

    \begin{Verbatim}[commandchars=\\\{\}]
{\color{incolor}In [{\color{incolor}124}]:} \PY{n+nb}{print}\PY{p}{(}\PY{n}{confusion\PYZus{}matrix}\PY{p}{(}\PY{n}{y\PYZus{}test}\PY{p}{,} \PY{n}{predict}\PY{p}{)}\PY{p}{)}
\end{Verbatim}


    \begin{Verbatim}[commandchars=\\\{\}]
[[3423    8]
 [  44 1025]]

    \end{Verbatim}


    % Add a bibliography block to the postdoc
    
    
    
    \end{document}
